\PassOptionsToPackage{utf8x}{inputenc}
\PassOptionsToPackage{pagebackref}{hyperref}
\usepackage[a-1b]{pdfx}
%\usepackage[a4paper,twoside,includeheadfoot,marginparwidth=20mm,marginparsep=2mm,left=20mm,right=30mm,top=30mm,bottom=30mm,headheight=16pt]{geometry}
\usepackage[a4paper,includeheadfoot,marginparwidth=20mm,marginparsep=2mm,left=20mm,right=30mm,top=30mm,bottom=30mm,headheight=16pt]{geometry}
\usepackage[T1]{fontenc}
\usepackage{amsmath,amssymb,amsthm}
\usepackage{fixmath}
\usepackage{bm}
%\usepackage{xcolor}
\usepackage{graphicx}
\usepackage{booktabs}
\usepackage{enumitem}
\usepackage{ifthen}
\usepackage[textsize=tiny]{todonotes}
\usepackage{listings}
\usepackage{lmodern}
\let\counterwithout\relax
\let\counterwithin\relax
\usepackage{chngcntr}
\usepackage{placeins}
\usepackage{multirow}
\usepackage[nocompress]{cite}
\usepackage{titlesec}
\titleformat{\section}{\normalfont\sffamily\Large\bfseries}{\thesection}{1em}{}
\titleformat{\subsection}{\normalfont\sffamily\large\bfseries}{\thesubsection}{1em}{}
\titleformat{\subsubsection}{\normalfont\sffamily\normalsize\bfseries}{\thesubsubsection}{1em}{}
\titleformat{\paragraph}[block]{\normalfont\sffamily\normalsize\bfseries}{\theparagraph}{1em}{}
\usepackage[titles]{tocloft}
\renewcommand{\cftsecpagefont}{\sffamily\bfseries}
\renewcommand{\cftsubsecpagefont}{\sffamily\small}
\renewcommand{\cftsubsubsecpagefont}{\sffamily\footnotesize}
\renewcommand{\cftsecfont}{\sffamily\bfseries}
\renewcommand{\cftsubsecfont}{\sffamily\small}
\renewcommand{\cftsubsubsecfont}{\sffamily\footnotesize}
\renewcommand{\cftfigfont}{\sffamily\small}
\renewcommand{\cftfigpagefont}{\sffamily\small}
\renewcommand{\cfttabfont}{\sffamily\small}
\renewcommand{\cfttabpagefont}{\sffamily\small}

\renewcommand{\listfigurename}{List of figures}
\renewcommand{\listtablename}{List of tables}
\renewcommand{\cftfigfont}{\sffamily\footnotesize}
\renewcommand{\cftfigpagefont}{\sffamily\footnotesize}
\renewcommand{\cfttabfont}{\sffamily\footnotesize}
\renewcommand{\cfttabpagefont}{\sffamily\footnotesize}
\renewcommand{\lstlistlistingname}{List of listings}
\makeatletter
\begingroup\let\newcounter\@gobble\let\setcounter\@gobbletwo
  \globaldefs\@ne \let\c@loldepth\@ne
  \newlistof{listings}{lol}{\lstlistlistingname}
\endgroup
\let\l@lstlisting\l@listings
\AtBeginDocument{\addtocontents{lol}{\protect\addvspace{10\p@}}}
\makeatother
\renewcommand{\lstlistoflistings}{\listoflistings}
\renewcommand{\cftlistingsfont}{\sffamily\footnotesize}
\renewcommand{\cftlistingspagefont}{\sffamily\footnotesize}
\renewcommand{\cftlistingsnumwidth}{2.3em}
\renewcommand{\cftlistingsindent}{1.5em}

\usepackage{fancyhdr}
\usepackage{subfig}

%\usepackage{hyperref}
%\hypersetup{allbordercolors={1 1 1},pdfstartview={Fit},pdfpagelayout={SinglePage},bookmarksdepth=subsubsection,pagebackref}
\hypersetup{allbordercolors={1 1 1},pdfstartview={Fit},pdfpagelayout={SinglePage},bookmarksdepth=subsubsection}

\graphicspath{{fig/}}

%\makeatletter
%\renewcommand\section{\clearpage{\pagestyle{empty}\cleardoublepage}\suppressfloats\@startsection {section}{1}{\z@}%
%  {-3.5ex \@plus -1ex \@minus -.2ex}%
%  {2.3ex \@plus.2ex}%
%  {\normalfont\sffamily\Large\bfseries}}
%\renewcommand\subsection{\@startsection{subsection}{2}{\z@}%
%  {-3.25ex\@plus -1ex \@minus -.2ex}%
%  {1.5ex \@plus .2ex}%
%  {\normalfont\sffamily\large\bfseries}}
%\renewcommand\subsubsection{\@startsection{subsubsection}{3}{\z@}%
%  {-3.25ex\@plus -1ex \@minus -.2ex}%
%  {1.5ex \@plus .2ex}%
%  {\normalfont\normalsize\sffamily\bfseries}}
%\makeatother

\makeatletter
\renewcommand\section{\suppressfloats\@startsection {section}{1}{\z@}%
  {-3.5ex \@plus -1ex \@minus -.2ex}%
  {2.3ex \@plus.2ex}%
  {\normalfont\sffamily\Large\bfseries}}
\renewcommand\subsection{\@startsection{subsection}{2}{\z@}%
  {-3.25ex\@plus -1ex \@minus -.2ex}%
  {1.5ex \@plus .2ex}%
  {\normalfont\sffamily\large\bfseries}}
\renewcommand\subsubsection{\@startsection{subsubsection}{3}{\z@}%
  {-3.25ex\@plus -1ex \@minus -.2ex}%
  {1.5ex \@plus .2ex}%
  {\normalfont\normalsize\sffamily\bfseries}}
\makeatother

\definecolor{codegreen}{rgb}{0,0.6,0}
\definecolor{codegray}{rgb}{0.5,0.5,0.5}
\definecolor{codepurple}{rgb}{0.58,0,0.82}
\definecolor{backcolour}{rgb}{0.95,0.95,0.92}
\definecolor{codeblue}{RGB}{0,160,238}
\definecolor{codeorange}{RGB}{217,83,25}
\lstdefinestyle{mystyle}{
    language=C++,
    %backgroundcolor=\color{backcolour},   
    commentstyle=\color{codeblue},
    keywordstyle=\color{codeorange},
    numberstyle=\tiny\color{codegray},
    %stringstyle=\color{codepurple},
    breakatwhitespace=false,         
    breaklines=true,                 
    captionpos=b,                    
    keepspaces=true,                 
    numbers=left,                    
    numbersep=5pt,                  
    showspaces=false,                
    showstringspaces=false,
    showtabs=false,                  
    tabsize=2,
    basicstyle={\footnotesize\ttfamily},
    xleftmargin=.1\textwidth, 
    xrightmargin=.1\textwidth,
    morekeywords={omp,offload,target,parallel,simd,critical,atomic,linear,declare,uniform,reduction,offload_transfer},
    framextopmargin=2pt,
    frame=tb,
    escapeinside={(|}{|)},
    moredelim=[is][\color{codeorange}]{|*}{*|}
}
\lstset{style=mystyle}


%%%%%%%% USER COMMANDS %%%%%%%%

\newcommand{\vect}[1]{\ensuremath{{\bm{#1}}}}

\newcommand{\va}{{\vect{a}}}
\newcommand{\vb}{{\vect{b}}}
\newcommand{\vc}{{\vect{c}}}
\newcommand{\vd}{{\vect{d}}}
\newcommand{\ve}{{\vect{e}}}
\newcommand{\vf}{{\vect{f}}}
\newcommand{\vg}{{\vect{g}}}
\newcommand{\vh}{{\vect{h}}}
\newcommand{\vi}{{\vect{i}}}
\newcommand{\vj}{{\vect{j}}}
\newcommand{\vk}{{\vect{k}}}
\newcommand{\vl}{{\vect{l}}}
\newcommand{\vm}{{\vect{m}}}
\newcommand{\vn}{{\vect{n}}}
\newcommand{\vo}{{\vect{o}}}
\newcommand{\vp}{{\vect{p}}}
\newcommand{\vq}{{\vect{q}}}
\newcommand{\vr}{{\vect{r}}}
\newcommand{\vs}{{\vect{s}}}
\newcommand{\vt}{{\vect{t}}}
\newcommand{\vu}{{\vect{u}}}
\newcommand{\vv}{{\vect{v}}}
\newcommand{\vw}{{\vect{w}}}
\newcommand{\vx}{{\vect{x}}}
\newcommand{\vy}{{\vect{y}}}
\newcommand{\vz}{{\vect{z}}}

\newcommand{\vA}{{\vect{A}}}
\newcommand{\vB}{{\vect{B}}}
\newcommand{\vC}{{\vect{C}}}
\newcommand{\vD}{{\vect{D}}}
\newcommand{\vE}{{\vect{E}}}
\newcommand{\vF}{{\vect{F}}}
\newcommand{\vG}{{\vect{G}}}
\newcommand{\vH}{{\vect{H}}}
\newcommand{\vI}{{\vect{I}}}
\newcommand{\vJ}{{\vect{J}}}
\newcommand{\vK}{{\vect{K}}}
\newcommand{\vL}{{\vect{L}}}
\newcommand{\vM}{{\vect{M}}}
\newcommand{\vN}{{\vect{N}}}
\newcommand{\vO}{{\vect{O}}}
\newcommand{\vP}{{\vect{P}}}
\newcommand{\vQ}{{\vect{Q}}}
\newcommand{\vR}{{\vect{R}}}
\newcommand{\vS}{{\vect{S}}}
\newcommand{\vT}{{\vect{T}}}
\newcommand{\vU}{{\vect{U}}}
\newcommand{\vV}{{\vect{V}}}
\newcommand{\vW}{{\vect{W}}}
\newcommand{\vX}{{\vect{X}}}
\newcommand{\vY}{{\vect{Y}}}
\newcommand{\vZ}{{\vect{Z}}}

\newcommand{\valpha}{{\vect{\alpha}}}
\newcommand{\veta}{{\vect{\eta}}}
\newcommand{\vpsi}{{\vect{\psi}}}
\newcommand{\vvarphi}{{\vect{\varphi}}}
\newcommand{\vmu}{{\vect{\mu}}}
\newcommand{\vnu}{{\vect{\nu}}}
\newcommand{\vxi}{{\vect{\xi}}}

\newcommand{\vzero}{{\vect{0}}}

\newcommand{\mat}[1]{\ensuremath{{\mathsf{#1}}}}

\newcommand{\matA}{{\mat{A}}}
\newcommand{\matB}{{\mat{B}}}
\newcommand{\matC}{{\mat{C}}}
\newcommand{\matD}{{\mat{D}}}
\newcommand{\matI}{{\mat{I}}}
\newcommand{\matK}{{\mat{K}}}
\newcommand{\matL}{{\mat{L}}}
\newcommand{\matM}{{\mat{M}}}
\newcommand{\matO}{{\mat{O}}}
\newcommand{\matP}{{\mat{P}}}
\newcommand{\matR}{{\mat{R}}}
\newcommand{\matS}{{\mat{S}}}
\newcommand{\matT}{{\mat{T}}}
\newcommand{\matV}{{\mat{V}}}
\newcommand{\matX}{{\mat{X}}}

\newcommand{\matXi}{{\mat{\Xi}}}

\newcommand{\map}[1]{\ensuremath{{\mathcal{#1}}}}

\newcommand{\calA}{{\ensuremath{\mathcal{A}}}}
\newcommand{\calB}{{\ensuremath{\mathcal{B}}}}
\newcommand{\calG}{{\ensuremath{\mathcal{G}}}}
\newcommand{\calK}{{\ensuremath{\mathcal{K}}}}
\newcommand{\calL}{{\ensuremath{\mathcal{L}}}}
\newcommand{\calO}{{\ensuremath{\mathcal{O}}}}
\newcommand{\calP}{{\ensuremath{\mathcal{P}}}}
\newcommand{\calR}{{\ensuremath{\mathcal{R}}}}

\newcommand{\dif}{\ensuremath{{\mathrm{d}}}}

\newcommand{\N}{{\mathbb{N}}}
\newcommand{\R}{{\mathbb{R}}}
\newcommand{\Z}{{\mathbb{Z}}}
\newcommand{\bbP}{{\mathbb{P}}}

\newcommand{\trans}{\ensuremath{{\mathsf{T}}}}

\newcommand{\atopf}[2]{\genfrac{}{}{0pt}{2}{#1}{#2}}

\DeclareMathOperator{\diver}{div}
\DeclareMathOperator{\vspan}{span}
\DeclareMathOperator{\dist}{dist}
\DeclareMathOperator{\dims}{dim}
\DeclareMathOperator{\diag}{diag}
\DeclareMathOperator{\meas}{meas}
\DeclareMathOperator{\supp}{supp}
\DeclareMathOperator{\sign}{sign}
\DeclareMathOperator{\curl}{\mathbf{curl}}
\DeclareMathOperator*{\esssup}{ess\:sup}
\DeclareMathOperator*{\argmin}{argmin}
\DeclareMathOperator{\erf}{erf}

\newcommand{\curlpo}{\ensuremath{\curl_{\partial\Omega}}}

%\newcommand{\todo}[1]{
%  \vspace{5 mm}\par \noindent
%  \marginpar{\textsc{Todo}}
%  \framebox{\begin{minipage}[c]{0.95 \linewidth}
%      \tt #1 \end{minipage}}\vspace{5 mm}\par
%}

%\newcommand{\comment}[1]{
%  \vspace{5 mm}\par \noindent
%  \marginpar{\textsc{Remark}}
%  \framebox{\begin{minipage}[c]{0.95 \linewidth}
%      \tt #1 \end{minipage}}\vspace{5 mm}\par
%}

\makeatletter
\renewcommand*\env@matrix[1][\arraystretch]{%
  \edef\arraystretch{#1}%
  \hskip -\arraycolsep
  \let\@ifnextchar\new@ifnextchar
  \array{*\c@MaxMatrixCols c}}
\makeatother

\theoremstyle{plain}
\newtheorem{theorem}{Theorem}[section]
\newtheorem{corollary}[theorem]{Corollary}
\newtheorem{lemma}[theorem]{Lemma}
\newtheorem{proposition}[theorem]{Proposition}
\theoremstyle{definition}
\newtheorem{definition}[theorem]{Definition}
\theoremstyle{remark}
\newtheorem{remark}[theorem]{Remark}

\numberwithin{equation}{section}
\numberwithin{figure}{section}
\numberwithin{table}{section}
\AtBeginDocument{\counterwithin{lstlisting}{section}}

\newcommand{\codeline}[1]{\mbox{\texttt{\frenchspacing #1}}}

\newcommand{\fref}[1]{Figure~\ref{#1}}
\newcommand{\sref}[1]{Section~\ref{#1}}

\overfullrule=6pt

\widowpenalty=10000
\clubpenalty=10000

\endinput